\begin{problem}{Epic Tournament}{standard input}{standard output}{2 seconds}{256 megabytes}

In celebration of International Women's Day, two best friends, Asmae and Meryem, decided to spend the day playing their favorite game, baby-foot. Asmae and Meryem were both skilled players, and they loved to play together in the school playground during their free time.

As they played, they heard a group of boys making fun of them. Laughing and saying stuff like "Girls can't play baby-foot". Besides having strong personalities, Asmae and Meryem knew they were good players, so they took their words as a challenge.

They decided to make a list of all the boys that doubted them, and when it was long enough, they made a plan for revenge.

On the same day, Asmae and Meryem organized a babyfoot tournament and invited all the boys on their list. They played hard with good teamwork, sometimes winning and sometimes losing, but always with honor. The two best friends came out proudly because they had proved their abilities.

\begin{center}
\includegraphics[width=16cm]{bbft.png}
\end{center}

As a final touch, Asmae and Meryem decided to print out their enemies' list and see if any names were repeated. If a name was repeated, they wanted to print the name and the number of times it appeared in the list, as a reminder that they had overcome their challenges and showed their doubters what they were truly capable of.

Given the initial list Asmae and Meryem made, you need to help them generate their new list. Besides each name print out how many times it appeared in the list. Each name should appear only once on the final list, in the same order they showed up on the first one.



\InputFile
The first line of the input contains an integer $n$ $(1 \leq n \leq 1000)$, size of the initial list.

Each of the next $n$ lines, contain a string $s$ of lowercase letters (of length $\leq 10$), names in the initial list.

\OutputFile
Print the final list, where each name appears only once with the number of its occurrences next to it, respecting the order in which they're given in the input.

\Example

\begin{example}
\exmpfile{example.01}{example.01.a}%
\end{example}

\end{problem}

